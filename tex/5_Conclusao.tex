Este capítulo mostra o que se alcançou com os objetivos. É feita uma análise crítica na seção~\ref{sec:analise} para verificar a completude dos objetivos propostos pelo trabalho. A seção~\ref{sec:limitacoes} aborda as limitações enfrentadas durante o desenvolvimento do trabalho. Neste trabalho foram alcançados resultados razoáveis para um trabalho que tinha como principal objetivo se familiarizar com a literatura e propor combinações de técnicas de \textit{deep learning} e compressão de imagens já existentes.

%%%%%%%%%%%%%%%%%%%%%%%%%
\section{Limitações do Trabalho}
\label{sec:limitacoes}
Modelos baseados em redes neurais para compressão de imagens precisam de muitas imagens para serem treinados e possuem custo computacional superior aos codecs clássicos para codificar e decodificar imagens. Esses modelos possuem o potencial de suprir uma necessidade crescente por algoritmos flexíveis de compressão com perdas. Entretanto, compressão com perdas é uma problema não diferenciável. Em particular, quantização é uma parte integral do \textit{pipeline} de compressão mas não é diferenciável, o que dificulta o trabalho de treinar redes neurais para esta tarefa.
%%%%%%%%%%%%%%%%%%%%%%%%%%
\section{Análise e Perspectivas Futuras}
\label{sec:analise}
As seguintes conclusões relacionadas as hipóteses~\ref{sec:hipotese} podem ser tomadas:
\begin{itemize}
    \item É possível notar que os modelos baseados em redes neurais são melhores do que os codecs clássicos para comprimir imagens com muito conteúdo de alta frequência em baixas taxas, conforme o esperado, visto que o Modelo 4 foi capaz de superar o \textbf{JPEG} e o \textbf{JPEG2000} para 47 imagens com muito conteúdo de alta frequência retiradas das bases Kodak~\cite{kodak} e CLIC~\cite{clic} nas métricas \acrshort{SSIM} e \acrshort{MS-SSIM}, além de ter superado o \acrshort{JPEG} na métrica \acrshort{PSNR} para essa base e para a base Kodak~\cite{kodak}.
    \item É possível concluir que o \textit{GZIP} não está comprimindo muito após a primeira iteração. Portanto, seria necessário desenvolver um codificador de entropia específico para esta tarefa ou alterar a arquitetura do modelo para tentar fazer com que os latentes binarizados sejam mais adequados para o codificador.
    \item Há um impacto significativo nos resultados ao usar imagens com muito conteúdo de alta frequência para treino dos modelos, conforme mostrado no capítulo anterior. 
    \item É possível notar um impacto razoável nos resultados com o uso de diferentes funções de custo, o que abre um espaço de exploração interessante para ser avaliado e explorado.
    \item Os modelos apresentados nesse trabalhos são flexíveis e podem ser facilmente aplicados em vídeos.
    \item Utiliza-se a mesma quantidade de bits para os mais diversos \textit{patches}, o que não é uma boa abordagem do ponto de vista de compressão visto que seria possível utilizar menos bits para \textit{patches} mais fáceis e mais bits para \textit{patches} mais difíceis. Seria interessante fazer com que o modelo fosse capaz de realizar uma alocação dinâmica de bits.
\end{itemize}

Todos os objetivos traçados foram atingidos e todas as questões levantadas nas hipóteses foram respondidas, entretanto ainda há muito espaço para novas soluções no problema abordado com vários desafios interessantes a serem enfrentados.
%%%%%%%%%%%%%%%%%%%%%%%%%%%
% \section{Trabalhos Futuros}
% \label{sec:futuro}
% Falar sobre a questão de usar mesma quantidade de bits para os mais diversos patches, o que não faz sentido do ponto de vista de compressão
% Ainda há muito espaço para novas soluções no âmbito do problema abordado, portanto foram definidas algumas atividades chaves para dar sequência ao trabalho. Será estudado formas de melhorar o desempenho dos modelos usando técnicas propostas nos trabalhos apresentados em~\ref{ae_coding} com técnicas comumente usadas em redes neurais convolucionais. Segue um cronograma em formato de Diagrama de Gantt que organiza as atividades do próximo semestre letivo.

% \begin{center}
% \begin{ganttchart}[y unit title=0.6cm,
%     y unit chart=1cm,
%     vgrid,hgrid,
%     title height=1,
%     bar/.style={draw,fill=cyan},
%     bar incomplete/.append style={fill=yellow!50},
%     bar height=0.7]{1}{17}
%     \gantttitle{Semanas}{17} \\
%     \gantttitlelist{1,...,17}{1} \\
%     \ganttbar{Estudar literatura}{1}{15} \\
%     \ganttbar{Implementar novo modelo}{4}{10} \\
%     \ganttbar{Melhorar novo modelo}{7}{13} \\
%     \ganttbar{Fazer testes}{7}{15} \\
%     \ganttbar{Escrever Monografia}{10}{17} \\
%     % \ganttbar[progress=40]{Fazer testes}{13}{14} \\
%     % relaçoẽs
%     %\ganttlink{elem0}{elem1}
%     \ganttlink{elem1}{elem2}
%     \ganttlink{elem2}{elem3}
%     \ganttlink{elem3}{elem2}
%     \ganttlink{elem3}{elem4}
%     \node[fill=white,draw] at ([yshift=-40pt, xshift=+68pt]current bounding box.south){Diagrama de Gantt};
% \end{ganttchart}   
% \end{center}
