Foram realizados vários experimentos com os modelos apresentados no capítulo anterior e com o \textit{JPEG} e \textit{JPEG2000} com o objetivo de avaliar o potencial de cada um deles nas bases de dados propostas. Os \textit{patches} usados em todos os experimentos realizados neste capítulo possuem 32 pixels de largura e altura. Para os testes em que o conjunto de treino e de teste são a mesma base, foram gerados dois subconjuntos disjuntos: um para treino e um para teste. Este último possui cerca de 10\% do tamanho total e, evidentemente, não faz parte do conjunto de treino. Os resultados do \textbf{JPEG} nas bases de teste utilizadas são apresentados na seção~\ref{res:jpeg} e os resultados dos modelos apresentados no capítulo anterior são apresentados nas seções~\ref{res:mod1}, \ref{res:mod2}, \ref{res:mod3}, \ref{res:mod4} e \ref{res:mod5}. Os três primeiros modelos apresentados possuem um único nível de resíduo.
%%%%%%%%%%%%%%%%%%%%%%%%%%%%%
\section{JPEG}
Nesta seção é apresentado o desempenho obtido pelo \textbf{JPEG} nas bases de teste. 
\label{res:jpeg}
\tabela{Tabela contendo médias obtidas pelo \textbf{JPEG} em cada uma das bases de teste utilizadas}{tab1:jpeg}{|l|l|l|l|l|l|}{\hline
\textbf{Bases}                                                                   & \textbf{\acrshort{BPP}} & \textbf{PSNR} & \textbf{SSIM} & \textbf{MSSIM} & \textbf{Quality} \\ \hline
\textbf{\begin{tabular}[c]{@{}l@{}}CLIC Mobile test\\ (patches 32)\end{tabular}} & 8            & 44.23         & 0.98          & 0.99           & 94.06            \\ \hline
\textbf{CLIC Mobile test}                                                        & 2            & 39.58         & 0.96          & 0.99           & 88.69            \\ \hline
\textbf{Kodak}                                                                   & 2            & 36.77         & 0.95          & 0.99           & 85.91            \\ \hline
\textbf{BD0}                                                                     & 8            & 57.85         & 0.99          & 0.99           & 99.18            \\ \hline
\textbf{BD1}                                                                     & 8            & 42.94         & 0.97          & 0.99           & 95.94            \\ \hline
\textbf{BD2}                                                                     & 8            & 32.31         & 0.95          & 0.99           & 83.69            \\ \hline
\textbf{BD3}                                                                     & 8            & 43.84         & 0.96          & 0.99           & 93.94            \\ \hline
\textbf{BD4}                                                                     & 8            & 36.72         & 0.96          & 0.99           & 89.68            \\ \hline}
%%%%%%%%%%%%%%%%%%%%%%%%%%%%%%%%%
\section{Modelo 1}
\label{res:mod1}
Primeiramente, o Modelo 1~[\refFig{conv_ae}] foi testado em todas as bases de dados usando \textit{learning rate} fixa durante todo o treinamento e o otimizador \textit{Adam}. Foi utilizada apenas uma época para treino em todos as avaliações realizadas. Os resultados são apresentados na Tabela~\ref{tab:mod1_bd}. Com estes mesmos hiperparâmetros para treino, foram realizados alguns testes na base de teste (nomeada como \textit{test}) do \acrshort{CLIC}~\cite{clic}. Alguns destes testes também usaram, em adição aos conjuntos de treino montados, os conjuntos de treino (\textit{train}) e validação (\textit{valid}) do \acrshort{CLIC}. Os resultados são apresentados na Tabela~\ref{tab:mod1_clic}.

\tabela{Tabela contendo o valor da \acrshort{PSNR}, em decíbeis, dos testes do Modelo 1 com o uso do otimizador \textit{Adam} e \textit{learning rate} fixa. As linhas denotam a base de treino utilizada. As colunas denotam as bases de teste usadas para avaliação do modelo. O índice ``todas'' se refere ao uso de todas as imagens de todas as bases \textbf{BD} para treino}{tab:mod1_bd}{|l|l|l|l|l|l|}{\hline
\textbf{\begin{tabular}[c]{@{}l@{}}Treino (linhas) x \\Teste (colunas)\end{tabular}} & \textbf{BD0}   & \textbf{BD1}   & \textbf{BD2}   & \textbf{BD3}   & \textbf{BD4}   \\ \hline
\textbf{BD0}                                                                          & 49.66          & 38.07          & 26.34          & 38.05          & 31.13          \\ \hline
\textbf{BD1}                                                                          & 47.57          & 44.08          & 34.54          & 42.66          & 38.69          \\ \hline
\textbf{BD2}                                                                          & \textbf{53.69} & \textbf{51.16} & \textbf{44.07} & \textbf{50.06} & \textbf{47.14} \\ \hline
\textbf{BD3}                                                                          & 50.62          & 47.88          & 39.76          & 46.59          & 43.25          \\ \hline
\textbf{BD4}                                                                          & 47.30          & 46.98          & 41.10          & 45.63          & 43.75          \\ \hline
\textbf{Todas}                                                                        & 46.77          & 46.35          & 43.94          & 45.88          & 45.08          \\ \hline}
\tabela{Tabela contendo o valor da \acrshort{PSNR}, em decíbeis, dos testes do Modelo 1 com o uso do otimizador \textit{Adam} e \textit{learning rate} fixa. As linhas denotam a base de treino utilizada. As colunas denotam as bases de teste usadas para avaliação do modelo. O índice ``todas'' se refere ao uso de todas as imagens de todas as bases \textbf{BD} para treino. O uso de $x+y$ denota o uso de todas as imagens do conjunto $x$ e do conjunto $y$ para treinamento}{tab:mod1_clic}{|l|l|}{
\hline
\textbf{Treino (linhas) x Teste (coluna)}                                                              & \textbf{\acrshort{CLIC} Mobile test} \\ \hline
\textbf{BD0}                                                                                            & 34.77                    \\ \hline
\textbf{BD1}                                                                                            & 44.11                    \\ \hline
\textbf{BD2}                                                                                            & 48.97                   \\ \hline
\textbf{BD3}                                                                                            & 45.34                   \\ \hline
\textbf{BD4}                                                                                            & 42.42                    \\ \hline
\textbf{Todas}                                                                                          & 51.27                     \\ \hline
\textbf{Todas + CLIC Mobile train}                                                                      & \textbf{55.78}            \\ \hline
\textbf{\begin{tabular}[c]{@{}l@{}}Todas + Clic Mobile train + \\ Clic Professional train\end{tabular}} & 47.26                    \\ \hline
\textbf{CLIC Mobile Train}                                                                              & 46.63                   \\ \hline
}
É interessante notar que o \textbf{BD2} é a melhor base de dados para treino, o que reforça os resultados encontrados por \textit{Toderici} em~\cite{FullResolution2017Toderici}. Outro resultado interessante obtido ocorre ao treinar na base de dados com menor entropia e testar na base com maior entropia. Pode-se notar que foi muito maléfico para a aprendizagem da rede treinar somente com exemplos fáceis para testar em exemplos difíceis. Posteriormente foram realizados alguns testes usando o método de atualização de \textit{learning rate} explicado no capítulo anterior. A política utilizada foi a exp\_range, pois foi a que obteve melhores resultados. Após a realização de vários testes, foi possível aumentar os \acrshort{dB} obtidos anteriormente de 44.08 e 44.07 ao treinar e testar no BD0 e BD1 para 50.81 e 47.83, respectivamente. Considerando que todos esses treinamentos referentes aos resultados apresentados nas~\refTabs{tab:mod1_bd}{tab:mod1_clic} foram executados utilizando uma única época, esta melhora pode ser considerada como a ``superconvergência'' apontada em~\cite{smith2017cyclical}. É interessante notar que, ao contrário do que foi observado por \textit{Leslie} neste artigo, o uso do otimizador \textit{Adam} com a exp\_range apresentou melhoras significativas para o Modelo 1.
%%%%%%%%%%%%%%%%%%%%%%%%%%%%%
\section{Modelo 2}
\label{res:mod2}
Para o Modelo 2~[\refFig{conv_ae_bin}], foram feitos testes nas bases \textbf{\acrshort{CLIC} Mobile}, \textbf{BD1} e \textbf{BD2}. Os resultados são apresentados na \refTab{tab1:mod2}. Conforme esperado, dentre as bases \textbf{BD1} e \textbf{BD2}, os piores resultados do \textit{JPEG} e do \textit{autoencoder} foram encontrados no \textbf{BD2} que é o com maior entropia (\acrshort{PNG} teve mais dificuldade para comprimir). Nota-se que a menor diferença proporcional entre o resultado do modelo e do \textit{JPEG} na métrica \acrshort{PSNR} se dá no \textbf{BD2}, o que reforça a observação feita em~\cite{Variable2016Toderici} de que em baixas taxas e resoluções espaciais, os artefatos blocantes do \textit{JPEG} (ruído causado pela perda de informação) se tornam mais comuns. Na~\refFig{patch} é mostrado um \textit{patch} reconstruído que obteve 36.69 \acrshort{dB} de \textit{PSNR}.
\tabela{Tabela contendo os resultados do Modelo 2 para as métricas visuais \acrshort{PSNR}, \acrshort{SSIM} e \acrshort{MS-SSIM} a uma taxa nominal de 8 bits por pixel}{tab1:mod2}{|l|l|l|l|l|l|}{\hline
\textbf{\begin{tabular}[c]{@{}l@{}}Bases de Treino e\\ Teste\end{tabular}} & \textbf{\acrshort{BPP}} & \textbf{PSNR} & \textbf{SSIM} & \textbf{MS-SSIM} & \textbf{Épocas} \\ \hline
\textbf{CLIC Mobile test}                                                       & 8            & 35.03         & 0.94          & 0.98           & 30              \\ \hline
\textbf{BD1}                                                               & 8            & 35.26         & 0.93          & 0.98           & 30              \\ \hline
\textbf{BD2}                                                               & 8            & 29.10         & 0.95          & 0.98           & 30              \\ \hline}
\figura[!htb]{patch}{Imagem original (esquerda) e \textit{patch} reconstruído pelo Modelo 2 (direita)}{patch}{width=\textwidth}
%%%%%%%%%%%%%%%%%%%%%%%%%%%%%
\section{Modelo 3}
\label{res:mod3}
Para o Modelo 3~[\refFig{toderici_model}], foram feitos testes nas bases \textbf{\acrshort{CLIC} Mobile}, \textbf{BD0}, \textbf{BD1}, \textbf{BD2}, \textbf{BD3}, \textbf{BD4} e \textbf{Kodak}. Os resultados são apresentados na~\refTab{tab:mod3}. Uma comparação com o \textit{JPEG} para diferentes taxas e métricas de distorção é apresentada na~\refFigs{plots_psnr}{plots_ssim}. Este modelo possui um único nível de resíduo.

\tabela{Tabela contendo os resultados do Modelo 3}{tab:mod3}{|l|l|l|l|l|}{\hline
\textbf{Bases}                                                                   & \textbf{\acrshort{BPP}} & \textbf{PSNR} & \textbf{SSIM} & \textbf{MS-SSIM} \\ \hline
\textbf{\begin{tabular}[c]{@{}l@{}}CLIC Mobile test\\ (patches 32)\end{tabular}} & 2            & 33.75         & 0.92          & 0.97            \\ \hline
\textbf{\begin{tabular}[c]{@{}l@{}}Kodak\\ (patches 32)\end{tabular}}            & 2            & 31.46         & 0.88          & 0.96            \\ \hline
\textbf{BD0}                                                                     & 2            & 40.24         & 0.97          & 0.99            \\ \hline
\textbf{BD1}                                                                     & 2            & 35.00         & 0.91          & 0.98            \\ \hline
\textbf{BD2}                                                                     & 2            & 27.53         & 0.91          & 0.97            \\ \hline
\textbf{BD3}                                                                     & 2            & 33.27         & 0.91          & 0.97            \\ \hline
\textbf{BD4}                                                                     & 2            & 30.16         & 0.90          & 0.97            \\ \hline}

\figura[!htb]{plots_ssim}{Comparação do Modelo 3 com o JPEG na métrica \acrshort{PSNR} em diferentes taxas}{plots_ssim}{width=0.7\textwidth}
\figura[!htb]{plots_psnr}{Comparação do Modelo 3 com o JPEG na métrica \acrshort{SSIM} em diferentes taxas}{plots_psnr}{width=0.7\textwidth}
%%%%%%%%%%%%%%%%%%%%%%%%%%%%%%%%%%%%%%%%%%%%%%%%%%%%%%%%%%
\section{Modelo 4}
\label{res:mod4}
Para este modelo foram usadas as bases \textbf{BD2}, \textbf{BD3} e \textbf{BD4} como treino e Kodak~\cite{kodak} para teste. O modelo possui 10 níveis de resíduos e foi treinado por 500 mil iterações. Para o JPEG e JPEG2000 são utilizados a melhor aproximação possível da \acrshort{BPP} obtida pela rede (network), ou seja, a melhor \textit{quality} que se adequa à \acrshort{BPP} desejada.

\tabela{Tabela contendo os valores da \acrshort{BPP} e PSNR (\acrshort{BPP}, PSNR) do Modelo 4, JPEG e JPEG2000}{tab:mod4_psnr}{|l|l|l|l|}{\hline
\textbf{Nível} & \textbf{Network} & \textbf{JPEG} & \textbf{JPEG2K} \\ \hline
\textbf{1}     & 0.10, 24.44      & 0.17, 21.52   & 0.10, 26.23     \\ \hline
\textbf{2}     & 0.23, 26.81      & 0.23, 24.51   & 0.23, 28.29     \\ \hline
\textbf{3}     & 0.35, 28.32      & 0.36, 27.53   & 0.35, 29.54     \\ \hline
\textbf{4}     & 0.48, 29.46      & 0.48, 29.08   & 0.47, 30.66     \\ \hline
\textbf{5}     & 0.60, 30.44      & 0.60, 30.23   & 0.60, 31.68     \\ \hline
\textbf{6}     & 0.73, 31.26      & 0.73, 31.14   & 0.72, 32.55     \\ \hline
\textbf{7}     & 0.85, 31.98      & 0.85, 31.92   & 0.85, 33.43     \\ \hline
\textbf{8}     & 0.98, 32.63      & 0.98, 32.60   & 0.97, 34.10     \\ \hline
\textbf{9}     & 1.10, 33.19      & 1.10, 33.21   & 1.10, 34.81     \\ \hline
\textbf{10}    & 1.23, 33.65      & 1.22, 33.72   & 1.21, \textbf{35.37}     \\ \hline}

\tabela{Tabela contendo os valores da \acrshort{BPP} e SSIM (\acrshort{BPP}, SSIM) do Modelo 4, JPEG e JPEG2000}{tab:mod4_ssim}{|l|l|l|l|}{\hline
\textbf{Nível} & \textbf{Network} & \textbf{JPEG} & \textbf{JPEG2K} \\ \hline
\textbf{1}     & 0.10, 0.65586       & 0.17, 0.56163    & 0.10, 0.69056      \\ \hline
\textbf{2}     & 0.23, 0.74829       & 0.23, 0.66125    & 0.23, 0.77412      \\ \hline
\textbf{3}     & 0.35, 0.80243       & 0.36, 0.76348    & 0.35, 0.81706      \\ \hline
\textbf{4}     & 0.48, 0.83860       & 0.48, 0.81317    & 0.47, 0.84603      \\ \hline
\textbf{5}     & 0.60, 0.86547       & 0.60, 0.84571    & 0.60, 0.86805      \\ \hline
\textbf{6}     & 0.73, 0.88592       & 0.73, 0.86819    & 0.72, 0.88374      \\ \hline
\textbf{7}     & 0.85, 0.90060       & 0.85, 0.88507    & 0.85, 0.89826      \\ \hline
\textbf{8}     & 0.98, 0.91252       & 0.98, 0.89808    & 0.97, 0.90900      \\ \hline
\textbf{9}     & 1.10, 0.92177       & 1.10, 0.90873    & 1.10, 0.91981      \\ \hline
\textbf{10}    & 1.23, \textbf{0.92873}       & 1.22, 0.91712    & 1.21, 0.92769      \\ \hline}

\tabela{Tabela contendo os valores da \acrshort{BPP} e MSSSIM (\acrshort{BPP}, MSSSIM) do Modelo 4, JPEG e JPEG2000}{tab:mod4_msssim}{|l|l|l|l|}{\hline
\textbf{Nível} & \textbf{Network} & \textbf{JPEG} & \textbf{JPEG2K} \\ \hline
\textbf{1}     & 0.10, 0.84270      & 0.17, 0.71702   & 0.10, 0.88245     \\ \hline
\textbf{2}     & 0.23, 0.92192      & 0.23, 0.82584   & 0.23, 0.93421     \\ \hline
\textbf{3}     & 0.35, 0.94893      & 0.36, 0.91058   & 0.35, 0.95471     \\ \hline
\textbf{4}     & 0.48, 0.96244      & 0.48, 0.94160   & 0.47, 0.96546     \\ \hline
\textbf{5}     & 0.60, 0.97054      & 0.60, 0.95843   & 0.60, 0.97269     \\ \hline
\textbf{6}     & 0.73, 0.97581      & 0.73, 0.96788   & 0.72, 0.97750     \\ \hline
\textbf{7}     & 0.85, 0.97993      & 0.85, 0.97420   & 0.85, 0.98118     \\ \hline
\textbf{8}     & 0.98, 0.98274      & 0.98, 0.97840   & 0.97, 0.98387     \\ \hline
\textbf{9}     & 1.10, 0.98502      & 1.10, 0.98158   & 1.10, 0.98630     \\ \hline
\textbf{10}    & 1.23, 0.98665      & 1.22, 0.98381   & 1.21, \textbf{0.98807}     \\ \hline}

Apesar da função de perda utilizada ser a \acrshort{MSE}, o modelo conseguiu obter melhores resultados do que os codecs na \acrshort{SSIM} e \acrshort{MS-SSIM}.
% Falar sobre o melhor resultado obtido na métrica SSIM e vantagem do modelo em frequências altas (kodim05) em contrapartida com frequências mais simples/baixas (kodim09)

\figura[!htb]{_mean_plot_psnr_10levels}{Comparação do Modelo 4 com o JPEG e JPEG2000 na métrica \acrshort{PSNR} em diferentes taxas para a base Kodak~\cite{kodak}}{_mean_plot_psnr_10levels}{width=0.6\textwidth}
\figura[!htb]{_mean_plot_ssim_10levels}{Comparação do Modelo 4 com o JPEG e JPEG2000 na métrica \acrshort{SSIM} em diferentes taxas para a base Kodak~\cite{kodak}}{_mean_plot_ssim_10levels}{width=0.6\textwidth}
\figura[!htb]{_mean_plot_msssim_10levels}{Comparação do Modelo 4 com o JPEG e JPEG2000 na métrica \acrshort{MS-SSIM} em diferentes taxas para a base Kodak~\cite{kodak}}{_mean_plot_msssim_10levels}{width=0.6\textwidth}

Nota-se que para imagens com muitos detalhes, com alto conteúdo de altas frequências, o modelo se sai melhor para todas as métricas visuais avaliadas. O que era esperado, visto que o JPEG e o JPEG2000 assumem que sinais de alta frequência não importam muito (assumem que maior parte energia da imagem estará contida em coeficientes de baixa frequência) e tende a priorizar que os coeficientes de baixa frequência tenham maior precisão ao quantizar.

\figura[!htb]{kodim05_10levels_comparison}{Imagem kodim05~\cite{kodak} original no canto superior esquerdo. Imagem comprimida pelo modelo no canto superior direito. Imagem do canto inferior esquerdo é a comprimida pelo JPEG e a última pelo JPEG2000}{kodim05_10levels_comparison}{width=\textwidth}

\figura[!htb]{kodim05_plot_psnr}{Comparação do Modelo 4 com o JPEG e JPEG2000 na métrica \acrshort{PSNR} em diferentes taxas para a imagem kodim05~\cite{kodak}}{kodim05_plot_psnr}{width=0.6\textwidth}
\figura[!htb]{kodim05_plot_ssim}{Comparação do Modelo 4 com o JPEG e JPEG2000 na métrica \acrshort{SSIM} em diferentes taxas para a imagem kodim05~\cite{kodak}}{kodim05_plot_ssim}{width=0.6\textwidth}
\figura[!htb]{kodim05_plot_msssim}{Comparação do Modelo 4 com o JPEG e JPEG2000 na métrica \acrshort{MS-SSIM} em diferentes taxas para a imagem kodim05~\cite{kodak}}{kodim05_plot_msssim}{width=0.6\textwidth}

\figura[!htb]{kodim09_10levels_comparison}{Imagem kodim09~\cite{kodak} original no canto superior esquerdo. Imagem comprimida pelo modelo no canto superior direito. Imagem do canto inferior esquerdo é a comprimida pelo JPEG e a última pelo JPEG2000}{kodim09_10levels_comparison}{width=0.8\textwidth}

\figura[!htb]{kodim09_plot_psnr}{Comparação do Modelo 4 com o JPEG e JPEG2000 na métrica \acrshort{PSNR} em diferentes taxas para a imagem kodim09~\cite{kodak}}{kodim09_plot_psnr}{width=0.6\textwidth}
\figura[!htb]{kodim09_plot_ssim}{Comparação do Modelo 4 com o JPEG e JPEG2000 na métrica \acrshort{SSIM} em diferentes taxas para a imagem kodim09~\cite{kodak}}{kodim09_plot_ssim}{width=0.6\textwidth}
\figura[!htb]{kodim09_plot_msssim}{Comparação do Modelo 4 com o JPEG e JPEG2000 na métrica \acrshort{MS-SSIM} em diferentes taxas para a imagem kodim09~\cite{kodak}}{kodim09_plot_msssim}{width=0.6\textwidth}

%%%%%%%%%%%%%%%%%%%%%%%%%%%%%%%%%%%%%%%%%%%%%%%%%%%%%%%%%%
\section{Modelo 5}
\ref{res:mod5}

