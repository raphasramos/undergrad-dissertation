Foram realizados vários experimentos com os modelos apresentados no capítulo anterior e com o \textit{JPEG} e \textit{JPEG2000} com o objetivo de avaliar o potencial de cada um deles nas bases de dados propostas. Os \textit{patches} usados em todos os experimentos realizados neste capítulo possuem 32 pixels de largura e altura. Para os testes em que o conjunto de treino e de teste são a mesma base, foram gerados dois subconjuntos disjuntos: um para treino e um para teste. Este último possui cerca de 10\% do tamanho total e, evidentemente, não faz parte do conjunto de treino. Os resultados do \textbf{JPEG} nas bases de teste utilizadas são apresentados na seção~\ref{res:jpeg} e os resultados dos modelos apresentados no capítulo anterior são apresentados nas seções~\ref{res:mod1}, \ref{res:mod2}, \ref{res:mod3}, \ref{res:mod4}. Os três primeiros modelos apresentados possuem um único nível de resíduo. Todas as taxas estão em bits por pixel.
%%%%%%%%%%%%%%%%%%%%%%%%%%%%%
\section{JPEG}
Nesta seção é apresentado o desempenho obtido pelo \textbf{JPEG} nas bases de teste. 
\label{res:jpeg}
\tabela{Tabela contendo médias obtidas pelo \textbf{JPEG} em cada uma das bases de teste utilizadas}{tab1:jpeg}{|l|l|l|l|l|l|}{\hline
\textbf{Bases}                                                                   & \textbf{Taxa} & \textbf{PSNR} & \textbf{SSIM} & \textbf{MSSIM} & \textbf{Quality} \\ \hline
\textbf{\begin{tabular}[c]{@{}l@{}}CLIC Mobile test\\ (patches 32)\end{tabular}} & 8            & 44.23         & 0.98          & 0.99           & 94.06            \\ \hline
\textbf{CLIC Mobile test}                                                        & 2            & 39.58         & 0.96          & 0.99           & 88.69            \\ \hline
\textbf{Kodak}                                                                   & 2            & 36.77         & 0.95          & 0.99           & 85.91            \\ \hline
\textbf{BD0}                                                                     & 8            & 57.85         & 0.99          & 0.99           & 99.18            \\ \hline
\textbf{BD1}                                                                     & 8            & 42.94         & 0.97          & 0.99           & 95.94            \\ \hline
\textbf{BD2}                                                                     & 8            & 32.31         & 0.95          & 0.99           & 83.69            \\ \hline
\textbf{BD3}                                                                     & 8            & 43.84         & 0.96          & 0.99           & 93.94            \\ \hline
\textbf{BD4}                                                                     & 8            & 36.72         & 0.96          & 0.99           & 89.68            \\ \hline}
%%%%%%%%%%%%%%%%%%%%%%%%%%%%%%%%%
\section{Modelo 1}
\label{res:mod1}
Primeiramente, o Modelo 1~[\refFig{conv_ae}] foi testado em todas as bases de dados usando \textit{learning rate} fixa durante todo o treinamento e o otimizador \textit{Adam}. Foi utilizada apenas uma época para treino em todos as avaliações realizadas. Os resultados são apresentados na Tabela~\ref{tab:mod1_bd}. Com estes mesmos hiperparâmetros para treino, foram realizados alguns testes na base de teste (nomeada como \textit{test}) do \acrshort{CLIC}~\cite{clic}. Alguns destes testes também usaram, em adição aos conjuntos de treino montados, os conjuntos de treino (\textit{train}) e validação (\textit{valid}) do \acrshort{CLIC}. Os resultados são apresentados na Tabela~\ref{tab:mod1_clic}.

\tabela{Tabela contendo o valor da \acrshort{PSNR}, em decíbeis, dos testes do Modelo 1 com o uso do otimizador \textit{Adam} e \textit{learning rate} fixa. As linhas denotam a base de treino utilizada. As colunas denotam as bases de teste usadas para avaliação do modelo. O índice ``todas'' se refere ao uso de todas as imagens de todas as bases \textbf{BD} para treino}{tab:mod1_bd}{|l|l|l|l|l|l|}{\hline
\textbf{\begin{tabular}[c]{@{}l@{}}Treino (linhas) x \\Teste (colunas)\end{tabular}} & \textbf{BD0}   & \textbf{BD1}   & \textbf{BD2}   & \textbf{BD3}   & \textbf{BD4}   \\ \hline
\textbf{BD0}                                                                          & 49.66          & 38.07          & 26.34          & 38.05          & 31.13          \\ \hline
\textbf{BD1}                                                                          & 47.57          & 44.08          & 34.54          & 42.66          & 38.69          \\ \hline
\textbf{BD2}                                                                          & \textbf{53.69} & \textbf{51.16} & \textbf{44.07} & \textbf{50.06} & \textbf{47.14} \\ \hline
\textbf{BD3}                                                                          & 50.62          & 47.88          & 39.76          & 46.59          & 43.25          \\ \hline
\textbf{BD4}                                                                          & 47.30          & 46.98          & 41.10          & 45.63          & 43.75          \\ \hline
\textbf{Todas}                                                                        & 46.77          & 46.35          & 43.94          & 45.88          & 45.08          \\ \hline}
\tabela{Tabela contendo o valor da \acrshort{PSNR}, em decíbeis, dos testes do Modelo 1 com o uso do otimizador \textit{Adam} e \textit{learning rate} fixa. As linhas denotam a base de treino utilizada. As colunas denotam as bases de teste usadas para avaliação do modelo. O índice ``todas'' se refere ao uso de todas as imagens de todas as bases \textbf{BD} para treino. O uso de $x+y$ denota o uso de todas as imagens do conjunto $x$ e do conjunto $y$ para treinamento}{tab:mod1_clic}{|l|l|}{
\hline
\textbf{Treino (linhas) x Teste (coluna)}                                                              & \textbf{\acrshort{CLIC} Mobile test} \\ \hline
\textbf{BD0}                                                                                            & 34.77                    \\ \hline
\textbf{BD1}                                                                                            & 44.11                    \\ \hline
\textbf{BD2}                                                                                            & 48.97                   \\ \hline
\textbf{BD3}                                                                                            & 45.34                   \\ \hline
\textbf{BD4}                                                                                            & 42.42                    \\ \hline
\textbf{Todas}                                                                                          & 51.27                     \\ \hline
\textbf{Todas + CLIC Mobile train}                                                                      & \textbf{55.78}            \\ \hline
\textbf{\begin{tabular}[c]{@{}l@{}}Todas + Clic Mobile train + \\ Clic Professional train\end{tabular}} & 47.26                    \\ \hline
\textbf{CLIC Mobile Train}                                                                              & 46.63                   \\ \hline
}
É interessante notar que o \textbf{BD2} é a melhor base de dados para treino, o que reforça os resultados encontrados por \textit{Toderici} em~\cite{FullResolution2017Toderici}. Outro resultado interessante obtido ocorre ao treinar na base de dados com menor entropia e testar na base com maior entropia. Pode-se notar que foi muito maléfico para a aprendizagem da rede treinar somente com exemplos fáceis para testar em exemplos difíceis. Posteriormente foram realizados alguns testes usando o método de atualização de \textit{learning rate} explicado no capítulo anterior. A política utilizada foi a exp\_range, pois foi a que obteve melhores resultados. Após a realização de vários testes, foi possível aumentar os \acrshort{dB} obtidos anteriormente de 44.08 e 44.07 ao treinar e testar no BD0 e BD1 para 50.81 e 47.83, respectivamente. Considerando que todos esses treinamentos referentes aos resultados apresentados nas~\refTabs{tab:mod1_bd}{tab:mod1_clic} foram executados utilizando uma única época, esta melhora pode ser considerada como a ``superconvergência'' apontada em~\cite{smith2017cyclical}. É interessante notar que, ao contrário do que foi observado por \textit{Leslie} neste artigo, o uso do otimizador \textit{Adam} com a exp\_range apresentou melhoras significativas para o Modelo 1.
%%%%%%%%%%%%%%%%%%%%%%%%%%%%%
\section{Modelo 2}
\label{res:mod2}
Para o Modelo 2~[\refFig{conv_ae_bin}], foram feitos testes nas bases \textbf{\acrshort{CLIC} Mobile}, \textbf{BD1} e \textbf{BD2}. Os resultados são apresentados na \refTab{tab1:mod2}. Conforme esperado, dentre as bases \textbf{BD1} e \textbf{BD2}, os piores resultados do \textit{JPEG} e do \textit{autoencoder} foram encontrados no \textbf{BD2} que é o com maior entropia (\acrshort{PNG} teve mais dificuldade para comprimir). Nota-se que a menor diferença proporcional entre o resultado do modelo e do \textit{JPEG} na métrica \acrshort{PSNR} se dá no \textbf{BD2}, o que reforça a observação feita em~\cite{Variable2016Toderici} de que em baixas taxas e resoluções espaciais, os artefatos blocantes do \textit{JPEG} (ruído causado pela perda de informação) se tornam mais comuns. Na~\refFig{patch} é mostrado um \textit{patch} reconstruído que obteve 36.69 \acrshort{dB} de \textit{PSNR}.
\tabela{Tabela contendo os resultados do Modelo 2 para as métricas visuais \acrshort{PSNR}, \acrshort{SSIM} e \acrshort{MS-SSIM} a uma taxa nominal de 8 bits por pixel}{tab1:mod2}{|l|l|l|l|l|l|}{\hline
\textbf{\begin{tabular}[c]{@{}l@{}}Bases de Treino e\\ Teste\end{tabular}} & \textbf{\acrshort{BPP}} & \textbf{PSNR} & \textbf{SSIM} & \textbf{MS-SSIM} & \textbf{Épocas} \\ \hline
\textbf{CLIC Mobile test}                                                       & 8            & 35.03         & 0.94          & 0.98           & 30              \\ \hline
\textbf{BD1}                                                               & 8            & 35.26         & 0.93          & 0.98           & 30              \\ \hline
\textbf{BD2}                                                               & 8            & 29.10         & 0.95          & 0.98           & 30              \\ \hline}
\figura[!htbp]{patch}{Imagem original (esquerda) e \textit{patch} reconstruído pelo Modelo 2 (direita)}{patch}{width=\textwidth}
%%%%%%%%%%%%%%%%%%%%%%%%%%%%%
\section{Modelo 3}
\label{res:mod3}
Para o Modelo 3, foram feitos testes nas bases \textbf{\acrshort{CLIC} Mobile}, \textbf{BD0}, \textbf{BD1}, \textbf{BD2}, \textbf{BD3}, \textbf{BD4} e \textbf{Kodak}. Os resultados são apresentados na~\refTab{tab:mod3}. Uma comparação com o \textit{JPEG} para diferentes taxas e métricas de distorção é apresentada na~\refFigs{plots_psnr}{plots_ssim}. Este modelo possui um único nível de resíduo.

\tabela{Tabela contendo os resultados do Modelo 3}{tab:mod3}{|l|l|l|l|l|}{\hline
\textbf{Bases}                                                                   & \textbf{Taxa} & \textbf{PSNR} & \textbf{SSIM} & \textbf{MS-SSIM} \\ \hline
\textbf{\begin{tabular}[c]{@{}l@{}}CLIC Mobile test\\ (patches 32)\end{tabular}} & 2            & 33.75         & 0.92          & 0.97            \\ \hline
\textbf{\begin{tabular}[c]{@{}l@{}}Kodak\\ (patches 32)\end{tabular}}            & 2            & 31.46         & 0.88          & 0.96            \\ \hline
\textbf{BD0}                                                                     & 2            & 40.24         & 0.97          & 0.99            \\ \hline
\textbf{BD1}                                                                     & 2            & 35.00         & 0.91          & 0.98            \\ \hline
\textbf{BD2}                                                                     & 2            & 27.53         & 0.91          & 0.97            \\ \hline
\textbf{BD3}                                                                     & 2            & 33.27         & 0.91          & 0.97            \\ \hline
\textbf{BD4}                                                                     & 2            & 30.16         & 0.90          & 0.97            \\ \hline}

\figura[!htbp]{plots_ssim}{Comparação do Modelo 3 com o JPEG na métrica \acrshort{PSNR} em diferentes taxas}{plots_ssim}{width=0.7\textwidth}
\figura[!htbp]{plots_psnr}{Comparação do Modelo 3 com o JPEG na métrica \acrshort{SSIM} em diferentes taxas}{plots_psnr}{width=0.7\textwidth}
%%%%%%%%%%%%%%%%%%%%%%%%%%%%%%%%%%%%%%%%%%%%%%%%%%%%%%%%%%
\section{Modelo 4}
\label{res:mod4}
Para este modelo foram usadas as bases \textbf{BD2}, \textbf{BD3} e \textbf{BD4} como treino. O modelo foi testado em duas bases: Kodak~\cite{kodak} e~\cite{clic}. O modelo possui 10 níveis de resíduos e foi treinado por 500 mil iterações. Os resultados referentes à base Kodak, são mostrados nas~\refFigs{_mean_plot_psnr_10levels}{_mean_plot_msssim_10levels} e sumarizados nas~\refTabs{tab:mod4_psnr}{tab:mod4_ssim}. Os resultados referentes às 47 imagens com muito conteúdo de alta frequência são mostrados nas~\refFigs{_mean_plot_psnr_hf_10levels}{_mean_plot_msssim_hf_10levels} e sumarizados nas~\refTabs{tab:mod4_hf_psnr}{tab:mod4_hf_msssim}.

\tabela{Tabela contendo os valores da taxa (\acrshort{BPP}) e PSNR (decíbeis) do Modelo 4, JPEG e JPEG2000 para a base~\cite{kodak}}{tab:mod4_psnr}{|c |c c |c c|c c| c c| }{\cline{2-9}
\multicolumn{1}{c}{} & \multicolumn{2}{|c|}{Network} & \multicolumn{2}{c|}{Network + GZIP} & \multicolumn{2}{c|}{JPEG} & \multicolumn{2}{c|}{JPEG 2000} \\ \hline
  \textbf{Nível} & Taxa & PSNR & Taxa & PSNR & Taxa & PSNR & Taxa & PSNR \\ \hline	
\textbf{1}     & $0.125$ & $24.44$ & $0.10$ & $24.44$ & $0.17$ & $21.52$ & $0.10$ & $26.23$     \\ 
\textbf{2}     & $0.250$ & $26.81$ & $0.23$ & $26.81$ & $0.23$ & $24.51$ & $0.23$ & $28.29$     \\ 
\textbf{3}     & $0.375$ & $28.32$ & $0.35$ & $28.32$ & $0.36$ & $27.53$ & $0.35$ & $29.54$     \\ 
\textbf{4}     & $0.500$ & $29.46$ & $0.48$ & $29.46$ & $0.48$ & $29.08$ & $0.47$ & $30.66$     \\ 
\textbf{5}     & $0.625$ & $30.44$ & $0.60$ & $30.44$ & $0.60$ & $30.23$ & $0.60$ & $31.68$     \\ 
\textbf{6}     & $0.750$ & $31.26$ & $0.73$ & $31.26$ & $0.73$ & $31.14$ & $0.72$ & $32.55$     \\ 
\textbf{7}     & $0.875$ & $31.98$ & $0.85$ & $31.98$ & $0.85$ & $31.92$ & $0.85$ & $33.43$     \\ 
\textbf{8}     & $1.000$ & $32.63$ & $0.98$ & $32.63$ & $0.98$ & $32.60$ & $0.97$ & $34.10$     \\ 
\textbf{9}     & $1.125$ & $33.19$ & $1.10$ & $33.19$ & $1.10$ & $33.21$ & $1.10$ & $34.81$     \\ 
\textbf{10}    & $1.250$ & $33.65$ & $1.23$ & $33.65$ & $1.22$ & $33.72$ & $1.21$ & \textbf{35.37}     \\ \hline}

\tabela{Tabela contendo os valores da taxa (\acrshort{BPP}) e SSIM do Modelo 4, JPEG e JPEG2000 para a base~\cite{kodak}}{tab:mod4_ssim}{|c |c c |c c|c c| c c| }{
\cline{2-9}
\multicolumn{1}{c}{} & \multicolumn{2}{|c|}{Network} & \multicolumn{2}{c|}{Network + GZIP} & \multicolumn{2}{c|}{JPEG} & \multicolumn{2}{c|}{JPEG 2000} \\ \hline
  \textbf{Nível} & Taxa & SSIM & Taxa & SSIM & Taxa & SSIM & Taxa & SSIM \\ \hline	
\textbf{1}     & $0.125$ &  $0.65586$ &  $0.10$  &  $0.65586$        &  $0.17$  &  $0.56163$     &  $0.10$  &  $0.69056$       \\ 
\textbf{2}     & $0.250$ &  $0.74829$ &  $0.23$  &  $0.74829$        &  $0.23$  &  $0.66125$     &  $0.23$  &  $0.77412$       \\ 
\textbf{3}     & $0.375$ &  $0.80243$ &  $0.35$  &  $0.80243$        &  $0.36$  &  $0.76348$     &  $0.35$  &  $0.81706$       \\ 
\textbf{4}     & $0.500$ &  $0.83860$ &  $0.48$  &  $0.83860$        &  $0.48$  &  $0.81317$     &  $0.47$  &  $0.84603$       \\ 
\textbf{5}     & $0.625$ &  $0.86547$ &  $0.60$  &  $0.86547$        &  $0.60$  &  $0.84571$     &  $0.60$  &  $0.86805$       \\ 
\textbf{6}     & $0.750$ &  $0.88592$ &  $0.73$  &  $0.88592$        &  $0.73$  &  $0.86819$     &  $0.72$  &  $0.88374$       \\ 
\textbf{7}     & $0.875$ &  $0.90060$ &  $0.85$  &  $0.90060$        &  $0.85$  &  $0.88507$     &  $0.85$  &  $0.89826$       \\ 
\textbf{8}     & $1.000$ &  $0.91252$ &  $0.98$  &  $0.91252$        &  $0.98$  &  $0.89808$     &  $0.97$  &  $0.90900$       \\ 
\textbf{9}     & $1.125$ &  $0.92177$ &  $1.10$  &  $0.92177$        &  $1.10$  &  $0.90873$     &  $1.10$  &  $0.91981$       \\ 
\textbf{10}    & $1.250$ &  $0.92873$ &  $1.23$  &  \textbf{0.92873}        &  $1.22$  &  $0.91712$     &  $1.21$  &  $0.92769$       \\ \hline}

\tabela{Tabela contendo os valores da taxa (\acrshort{BPP}) e MSSSIM do Modelo 4, JPEG e JPEG2000 para a base~\cite{kodak}}{tab:mod4_msssim}{|c |c c |c c|c c| c c| }{\cline{2-9}
\multicolumn{1}{c}{} & \multicolumn{2}{|c|}{Network} & \multicolumn{2}{c|}{Network + GZIP} & \multicolumn{2}{c|}{JPEG} & \multicolumn{2}{c|}{JPEG 2000} \\ \hline
  \textbf{Nível} & Taxa & MS & Taxa & MS & Taxa & MS & Taxa & MS \\ \hline	
  \textbf{1}     & $0.125$ &  $0.84270$  &  $0.10$  &  $0.84270$       &  $0.17$  &  $0.71702$    &  $0.10$  &  $0.88245$      \\ 
  \textbf{2}     & $0.250$ &  $0.92192$  &  $0.23$  &  $0.92192$       &  $0.23$  &  $0.82584$    &  $0.23$  &  $0.93421$      \\ 
  \textbf{3}     & $0.375$ &  $0.94893$  &  $0.35$  &  $0.94893$       &  $0.36$  &  $0.91058$    &  $0.35$  &  $0.95471$      \\ 
  \textbf{4}     & $0.500$ &  $0.96244$  &  $0.48$  &  $0.96244$       &  $0.48$  &  $0.94160$    &  $0.47$  &  $0.96546$      \\ 
  \textbf{5}     & $0.625$ &  $0.97054$  &  $0.60$  &  $0.97054$       &  $0.60$  &  $0.95843$    &  $0.60$  &  $0.97269$      \\ 
  \textbf{6}     & $0.750$ &  $0.97581$  &  $0.73$  &  $0.97581$       &  $0.73$  &  $0.96788$    &  $0.72$  &  $0.97750$      \\ 
  \textbf{7}     & $0.875$ &  $0.97993$  &  $0.85$  &  $0.97993$       &  $0.85$  &  $0.97420$    &  $0.85$  &  $0.98118$      \\ 
  \textbf{8}     & $1.000$ &  $0.98274$  &  $0.98$  &  $0.98274$       &  $0.98$  &  $0.97840$    &  $0.97$  &  $0.98387$      \\ 
  \textbf{9}     & $1.125$ &  $0.98502$  &  $1.10$  &  $0.98502$       &  $1.10$  &  $0.98158$    &  $1.10$  &  $0.98630$      \\ 
  \textbf{10}    & $1.250$ &  $0.98665$  &  $1.23$  &  $0.98665$       &  $1.22$  &  $0.98381$    &  $1.21$  &  \textbf{0.98807} 	 \\ \hline}

Apesar da função de perda utilizada ser a \acrshort{MSE}, o modelo conseguiu obter melhores resultados comparados aos codecs na \acrshort{SSIM} e \acrshort{MS-SSIM}.

\figura[!htbp]{_mean_plot_psnr_10levels}{Comparação do Modelo 4 com o JPEG e JPEG2000 na métrica \acrshort{PSNR} em diferentes taxas para a base Kodak~\cite{kodak}}{_mean_plot_psnr_10levels}{width=0.6\textwidth}
\figura[!htbp]{_mean_plot_ssim_10levels}{Comparação do Modelo 4 com o JPEG e JPEG2000 na métrica \acrshort{SSIM} em diferentes taxas para a base Kodak~\cite{kodak}}{_mean_plot_ssim_10levels}{width=0.6\textwidth}
\figura[!htbp]{_mean_plot_msssim_10levels}{Comparação do Modelo 4 com o JPEG e JPEG2000 na métrica \acrshort{MS-SSIM} em diferentes taxas para a base Kodak~\cite{kodak}}{_mean_plot_msssim_10levels}{width=0.6\textwidth}

Nota-se que para imagens (\refFigs{kodim05}{jason-leem-143987}) com muito conteúdo de alta frequência (muitos detalhes), o modelo se sai melhor para todas as métricas visuais avaliadas. O que era esperado, visto que o JPEG e o JPEG2000 assumem que sinais de alta frequência não importam muito (assumem que maior parte energia da imagem estará contida em coeficientes de baixa frequência) e faz com que os coeficientes de baixa frequência tenham maior precisão ao quantizar.

\figura[!htbp]{kodim05}{Imagem Kodim05~\cite{kodak}}{kodim05}{width=0.7\textwidth}
\figura[!htbp]{jason-leem-143987}{Imagem jason-leem~\cite{clic}}{jason-leem-143987}{width=0.7\textwidth}

\figura[!htbp]{kodim05_plot_psnr}{Comparação do Modelo 4 com o JPEG e JPEG2000 na métrica \acrshort{PSNR} em diferentes taxas para a imagem~\refFig{kodim05}}{kodim05_plot_psnr}{width=0.6\textwidth}
\figura[!htbp]{kodim05_plot_ssim}{Comparação do Modelo 4 com o JPEG e JPEG2000 na métrica \acrshort{SSIM} em diferentes taxas para a imagem~\refFig{kodim05}}{kodim05_plot_ssim}{width=0.6\textwidth}
\figura[!htbp]{kodim05_plot_msssim}{Comparação do Modelo 4 com o JPEG e JPEG2000 na métrica \acrshort{MS-SSIM} em diferentes taxas para a imagem~\refFig{kodim05}}{kodim05_plot_msssim}{width=0.6\textwidth}

\figura[!htbp]{jason-leem-143987_plot_psnr}{Comparação do Modelo 4 com o JPEG e JPEG2000 na métrica \acrshort{PSNR} em diferentes taxas para a imagem de~\cite{clic}}{jason-leem-143987_plot_psnr}{width=0.6\textwidth}
\figura[!htbp]{jason-leem-143987_plot_ssim}{Comparação do Modelo 4 com o JPEG e JPEG2000 na métrica \acrshort{SSIM} em diferentes taxas para a imagem de~\cite{clic}}{jason-leem-143987_plot_ssim}{width=0.6\textwidth}
\figura[!htbp]{jason-leem-143987_plot_msssim}{Comparação do Modelo 4 com o JPEG e JPEG2000 na métrica \acrshort{MS-SSIM} em diferentes taxas para a imagem de~\cite{clic}}{jason-leem-143987_plot_msssim}{width=0.6\textwidth}

Considerando os resultados anteriores obtidos para as imagens kodim05 e kodim09, o Modelo 4 também foi testado em um conjunto de 47 imagens com muito conteúdo de alta frequência retiradas da~\cite{kodak} e da~\cite{clic}. Os resultados são mostrados nas~\refFigs{_mean_plot_psnr_hf_10levels}{_mean_plot_msssim_hf_10levels} e sumarizados nas~\refTabs{tab:mod4_hf_psnr}{tab:mod4_hf_msssim}. 

\figura[!htbp]{_mean_plot_psnr_hf_10levels}{Comparação do Modelo 4 com o JPEG e JPEG2000 na métrica \acrshort{PSNR} em diferentes taxas para 47 imagens com muito conteúdo de alta frequência retiradas das bases~\cite{clic} e~\cite{kodak}}{_mean_plot_psnr_hf_10levels}{width=0.6\textwidth}
\figura[!htbp]{_mean_plot_ssim_hf_10levels}{Comparação do Modelo 4 com o JPEG e JPEG2000 na métrica \acrshort{SSIM} em diferentes taxas para 47 imagens com muito conteúdo de alta frequência retiradas das bases~\cite{clic} e~\cite{kodak}}{_mean_plot_ssim_hf_10levels}{width=0.6\textwidth}
\figura[!htbp]{_mean_plot_msssim_hf_10levels}{Comparação do Modelo 4 com o JPEG e JPEG2000 na métrica \acrshort{MS-SSIM} em diferentes taxas para 47 imagens com muito conteúdo de alta frequência retiradas das bases~\cite{clic} e~\cite{kodak}}{_mean_plot_msssim_hf_10levels}{width=0.6\textwidth}

\tabela{Tabela contendo os valores da taxa (\acrshort{BPP}) e PSNR do Modelo 4, JPEG e JPEG2000 para 47 imagens da base~\cite{kodak} e~\cite{clic} com muito conteúdo de alta frequência}{tab:mod4_hf_psnr}{|c |c c |c c|c c| c c| }{\cline{2-9}
\multicolumn{1}{c}{} & \multicolumn{2}{|c|}{Network} & \multicolumn{2}{c|}{Network + GZIP} & \multicolumn{2}{c|}{JPEG} & \multicolumn{2}{c|}{JPEG 2000} \\ \hline
  \textbf{Nível} & Taxa & PSNR & Taxa & PSNR & Taxa & PSNR & Taxa & PSNR \\ \hline	
  \textbf{1}     & $0.125$ & $24.69$         & $0.10$ & $24.69$      & $0.17$ & $21.51$   & $0.10$ & $26.63$     \\
  \textbf{2}     & $0.250$ & $27.23$         & $0.21$ & $27.23$      & $0.22$ & $24.64$   & $0.21$ & $28.59$     \\
  \textbf{3}     & $0.375$ & $28.72$         & $0.34$ & $28.72$      & $0.34$ & $27.63$   & $0.33$ & $29.70$     \\
  \textbf{4}     & $0.500$ & $29.77$         & $0.46$ & $29.77$      & $0.46$ & $28.98$   & $0.46$ & $30.54$     \\
  \textbf{5}     & $0.625$ & $30.64$         & $0.59$ & $30.64$      & $0.59$ & $29.90$   & $0.58$ & $31.31$     \\
  \textbf{6}     & $0.750$ & $31.38$         & $0.71$ & $31.38$      & $0.71$ & $30.64$   & $0.71$ & $31.99$     \\
  \textbf{7}     & $0.875$ & $32.01$         & $0.84$ & $32.01$      & $0.84$ & $31.22$   & $0.83$ & $32.66$     \\
  \textbf{8}     & $1.000$ & $32.57$         & $0.96$ & $32.57$      & $0.96$ & $31.75$   & $0.95$ & $33.23$     \\
  \textbf{9}     & $1.125$ & $33.06$         & $1.09$ & $33.06$      & $1.10$ & $32.23$   & $1.08$ & $33.81$     \\
  \textbf{10}    & $1.250$ & $33.45$         & $1.20$ & $33.45$      & $1.21$ & $32.67$   & $1.20$ & \textbf{34.30}     \\ \hline}

\tabela{Tabela contendo os valores da taxa (\acrshort{BPP}) e SSIM do Modelo 4, JPEG e JPEG2000 para 47 imagens da base~\cite{kodak} e~\cite{clic} com muito conteúdo de alta frequência}{tab:mod4_hf_ssim}{|c |c c |c c|c c| c c| }{\cline{2-9}
\multicolumn{1}{c}{} & \multicolumn{2}{|c|}{Network} & \multicolumn{2}{c|}{Network + GZIP} & \multicolumn{2}{c|}{JPEG} & \multicolumn{2}{c|}{JPEG 2000} \\ \hline
\textbf{Nível} & Taxa & SSIM & Taxa & SSIM & Taxa & SSIM & Taxa & SSIM \\ \hline	
\textbf{1}     & $0.125$ & $0.64562$ & $0.10$ & $0.64562$      & $0.17$ & $0.55499$   & $0.10$ & $0.68526$     \\
\textbf{2}     & $0.250$ & $0.74250$ & $0.21$ & $0.74250$      & $0.22$ & $0.64966$   & $0.21$ & $0.76202$     \\ 
\textbf{3}     & $0.375$ & $0.79480$ & $0.34$ & $0.79480$      & $0.34$ & $0.74767$   & $0.33$ & $0.80283$     \\ 
\textbf{4}     & $0.500$ & $0.82961$ & $0.46$ & $0.82961$      & $0.46$ & $0.79561$   & $0.46$ & $0.82961$     \\ 
\textbf{5}     & $0.625$ & $0.85558$ & $0.59$ & $0.85558$      & $0.59$ & $0.82607$   & $0.58$ & $0.84989$     \\ 
\textbf{6}     & $0.750$ & $0.87530$ & $0.71$ & $0.87530$      & $0.71$ & $0.84832$   & $0.71$ & $0.86553$     \\ 
\textbf{7}     & $0.875$ & $0.88973$ & $0.84$ & $0.88973$      & $0.84$ & $0.86414$   & $0.83$ & $0.87862$     \\ 
\textbf{8}     & $1.000$ & $0.90168$ & $0.96$ & $0.90168$      & $0.96$ & $0.87722$   & $0.95$ & $0.89020$     \\ 
\textbf{9}     & $1.125$ & $0.91092$ & $1.09$ & $0.91092$      & $1.10$ & $0.88800$   & $1.08$ & $0.90109$     \\ 
\textbf{10}    & $1.250$ & $0.91784$  & $1.20$ & \textbf{0.91784}      & $1.21$ & $0.89648$   & $1.20$ & $0.90973$  \\ \hline}

\tabela{Tabela contendo os valores da taxa (\acrshort{BPP}) e MSSSIM do Modelo 4, JPEG e JPEG2000 para 47 imagens da base~\cite{kodak} e~\cite{clic} com muito conteúdo de alta frequência}{tab:mod4_hf_msssim}{|c |c c |c c|c c| c c| }{\cline{2-9}
\multicolumn{1}{c}{} & \multicolumn{2}{|c|}{Network} & \multicolumn{2}{c|}{Network + GZIP} & \multicolumn{2}{c|}{JPEG} &\multicolumn{2}{c|}{JPEG 2000} \\ \hline
\textbf{Nível} & Taxa & MS & Taxa & MS & Taxa & MS & Taxa & MS \\ \hline	
\textbf{1}     & 0.10 & 0.84156     & 0.10 & 0.84156      & 0.17 & 0.72882   & 0.10 & 0.87953     \\ 
\textbf{2}     & 0.21 & 0.92131     & 0.21 & 0.92131      & 0.22 & 0.82462   & 0.21 & 0.92713     \\ 
\textbf{3}     & 0.34 & 0.94802     & 0.34 & 0.94802      & 0.34 & 0.90454   & 0.33 & 0.94699     \\ 
\textbf{4}     & 0.46 & 0.96112     & 0.46 & 0.96112      & 0.46 & 0.93426   & 0.46 & 0.95877     \\ 
\textbf{5}     & 0.59 & 0.96897     & 0.59 & 0.96897      & 0.59 & 0.95006   & 0.58 & 0.96628     \\ 
\textbf{6}     & 0.71 & 0.97408     & 0.71 & 0.97408      & 0.71 & 0.96010   & 0.71 & 0.97148     \\ 
\textbf{7}     & 0.84 & 0.97806     & 0.84 & 0.97806      & 0.84 & 0.96655   & 0.83 & 0.97552     \\ 
\textbf{8}     & 0.96 & 0.98088     & 0.96 & 0.98088      & 0.96 & 0.97134   & 0.95 & 0.97863     \\ 
\textbf{9}     & 1.09 & 0.98315     & 1.09 & 0.98315      & 1.10 & 0.97483   & 1.08 & 0.98151     \\ 
\textbf{10}    & 1.20 & 0.98477     & 1.20 & \textbf{0.98477}      & 1.21 & 0.97746   & 1.20 & 0.98356 \\ \hline}

Pelas~\refFigs{kodim05_3_net}{kodim05_3_jpeg2k} é possível notar o que as~\refFigs{kodim05_plot_psnr}{kodim05_plot_msssim} já haviam mostrado: a reconstrução do Modelo 4 é visualmente superior para a~\refFig{kodim05} quando comparada com o \textbf{JPEG} e \textbf{JPEG2000}, mesmo com uma taxa menor. Essa taxa corresponde ao quarto nível de resíduo para o Modelo 4.

\figura[!htbp]{kodim05_3_net}{Imagem~\refFig{kodim05} reconstruída pelo Modelo 4 à uma taxa de 0.47 bits por pixel.}{kodim05_3_net}{width=0.7\textwidth}
\figura[!htbp]{kodim05_3_jpeg}{Imagem~\refFig{kodim05} reconstruída pelo \textbf{JPEG} à uma taxa de 0.50 bits por pixel.}{kodim05_3_jpeg}{width=0.7\textwidth}
\figura[!htbp]{kodim05_3_jpeg2k}{Imagem~\refFig{kodim05} reconstruída pelo \textbf{JPEG2000} à uma taxa de 0.50 bits por pixel}{kodim05_3_jpeg2k}{width=0.7\textwidth}

Para taxas muito baixas é possível notar uma grande dificuldade do \textbf{JPEG} em comprimir para a~\refFig{kodim05} mesmo com uma taxa maior, conforme mostra as~\refFigs{kodim05_0_net}{kodim05_0_jpeg2k}. Essa taxa corresponde ao primeiro nível de resíduo para o Modelo 4.

\figura[!htbp]{kodim05_0_net}{Imagem~\refFig{kodim05} reconstruída pelo Modelo 4 à uma taxa de 0.12 bits por pixel.}{kodim05_0_net}{width=0.7\textwidth}
\figura[!htbp]{kodim05_0_jpeg}{Imagem~\refFig{kodim05} reconstruída pelo \textbf{JPEG} à uma taxa de 0.21 bits por pixel.}{kodim05_0_jpeg}{width=0.7\textwidth}
\figura[!htbp]{kodim05_0_jpeg2k}{Imagem~\refFig{kodim05} reconstruída pelo \textbf{JPEG2000} à uma taxa de 0.12 bits por pixel}{kodim05_0_jpeg2k}{width=0.7\textwidth}
%%%%%%%%%%%%%%%%%%%%%%%%%%%%%%%%%%%%%%%%%%%%%%%%%%%%%%%%%%
\section{Modelos 5, 6 e 7}
\label{res:6levels}
Esses três modelos foram treinados com 150000 iterações nas bases \textbf{BD2} e \textbf{BD4}. Foram utilizados 6 níveis de resíduos. 

% Colocar os plots de comparação dos modelos treinados usando mse, ssim e pseudo-variacional loss.
A função de loss utilizada no Modelo x é \equacao{loss}{Ke^{(-0.07r)}Q} onde $K = 3.5 \times 10^{-7}$ e $Q$ é a quantidade de bits 1 no latente binarizado. Com esta função de \textit{loss} há uma penalização na ocorrência do bit 1 o que leva a uma diminuição da entropia do latente binarizado. Espera-se que o codificador de entropia seja mais eficaz dessa forma.

A função de loss utilizada no Modelo x+1 é \equacao{loss_ssim}{1 - SSIM(x, \hat{x})}, visto que a \acrshort{SSIM} é um valor entre 0 e 1 diretamente proporcional à qualidade visual da imagem reconstruída a partir do latente binarizado $\hat{x}$.  
%%%%%%%%%%%%%%%%%%%%%%%%%%%%%%%%%%%%%%%%%%%%%%%%%%%%%%%%%%
\section{Modelos 8, 9 e 10}
% Colocar os plots de comparação dos modelos treinados usando diferentes ranges de entrada

%%%%%%%%%%%%%%%%%%%%%%%%%%%%%%%%%%%%%%%%%%%%%%%%%%%%%%%%%%
% Comparar ganhos obtidos pelo GZIP nos modelos com diferentes funções de loss 
\section{Ganhos obtidos pelo GZIP}
\label{res:gzip}
Nas seguintes subseções será analisado o ganho na taxa obtido ao usar o codificador de entropia \textit{gzip}.
\subsection{Modelo 4}
A~\refFig{gain_gzip_mod4} mostra o ganho percentual médio por nível do Modelo 4~[\ref{res:mod4}] para a base da Kodak~\cite{kodak}. Nota-se um decaímento exponencial no ganho. Considerando o funcionamento do modelo com resíduos, isto sugere uma especialização e robustez cada vez maior nos \textit{encoders} dos níveis mais superiores ao explorar estruturas dos dados e comprimir informação nos latentes. Assim, os \textit{bitstreams} gerados a partir dos latentes se tornam cada vez mais complexos (os resíduos possuirão muita variância nos valores) e com menos redundâncias fazendo com que o \textit{gzip} não consiga explorar redundâncias bem ou que não haja muitas redundâncias. 

Isto acontece na maior parte das imagens, principalmente nas que possui muito conteúdo de alta frequência e onde há muita informação que, consequentemente, geram latentes muito complexos, o que faz com que os \textit{bitstreams} contenham muita informação já que os resíduos dos \textit{patches} contém muita informação e geram muitas ativações no modelo.

\figura[!htbp]{gain_gzip_mod4}{Ganho percentual médio na taxa por nível ao usar o codificador de entropia \textit{gzip} nos \textit{bitstreams} de cada nível para a base Kodak~\cite{kodak}}{gain_gzip_mod4}{width=\textwidth}
% \figura[!htbp]{gain_gzip_mod4_hf}{Ganho percentual médio na taxa por nível ao usar o codificador de entropia \textit{gzip} nos \textit{bitstreams} de cada nível para 47 imagens com muito conteúdo de alta frequência da base Kodak~\cite{kodak} e~\cite{clic}}{gain_gzip_mod4_hf}{width=\textwidth}

Na~\refFig{gain_gzip_kodim05} é possível notar um ganho praticamente insignificante (e até negativo, prejudicial) ao usar o \textit{gzip} para a~\refFig{kodim05}, enquanto na~\refFig{gain_gzip_kodim20} nota-se um ganho mais significativo com o \textit{gzip} mesmo nos níveis mais superiores. Considerando que a~\refFig{kodim05} possui muito conteúdo de alta frequência, os \textit{bitstreams} gerados pelo modelo se tornam muito complexos. Enquanto, a~\refFig{kodim20} possui um céu que faz parte da maior parte da imagem. É possível que os resíduos desses \textit{patches} se tornem pequenos muito rápido de modo que o \textit{gzip} é capaz de explorar bem esta grande quantidade de zeros.

\figura[!htbp]{kodim20}{Imagem Kodim20~\cite{kodak}}{kodim20}{width=0.7\textwidth}
\figura[!htbp]{gain_gzip_kodim05}{Ganho percentual na taxa por nível ao usar o codificador de entropia \textit{gzip} nos \textit{bitstreams} de cada nível para a~\refFig{kodim05}}{gain_gzip_kodim05}{width=0.7\textwidth}
\figura[!htbp]{gain_gzip_kodim20}{Ganho percentual na taxa por nível ao usar o codificador de entropia \textit{gzip} nos \textit{bitstreams} de cada nível para a~\refFig{kodim20}}{gain_gzip_kodim20}{width=0.7\textwidth}
