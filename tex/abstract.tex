The resources required to store and transmit images are huge, making their compression essential. All the efforts made in classical image compression algorithms address the problem from an empiric point of view: experts develop several heuristics to reduce the amount of information needed to represent images by exploiting imperfections in the human visual system. This way, it is possible to reconstruct them without much perceptible information loss.

The success of deep convolutional neural networks (CNNs) in computer vision application has been inspiring researchers from the image compression community to try to develop algorithms that learn from data, rather than relying on expert knowledge. Autoencoder networks are used in these algorithms, since dimensionality reduction is part of their operation. So far, these algorithms have not lead to a significant improvement over classical codecs.

The objective of the present work is to study and explore neural network solutions to the challenge of compressing images, aiming to propose a method that can be competitive to classical codecs in some scenarios. To achieve this goal, the classical compression methods JPEG and JPEG2000 are evaluated in databases commonly used for this type of problem, as they were the first two methods widely used. Next, some experiments on encoder-decoder image compression framework are presented.