

Os recursos necessários para armazenar e transmitir imagens são imensos, o que torna a sua compressão necessária. Todos os esforços feitos em algoritmos de compressão de imagens clássicos abordam o problema de compressão de um ponto de vista empírico: humanos desenvolvem várias heurísticas para reduzir a quantidade de informação necessária para representar a imagem explorando imperfeições no sistema visual humano, de modo que seja possível reconstruí-la sem muita perda de informação. Compressão de imagens usando redes neurais tem sido uma área ativa de pesquisa em tempos recentes com vários desafios a serem enfrentados para que essas técnicas sejam competitivas com os codificadores clássicos.

O objetivo do presente trabalho é estudar e explorar soluções para o desafio de comprimir imagens. Para isso, foi analisado o método de compressão clássico \textit{JPEG} e métodos usando \textit{autoencoders} convolucionais. Esses métodos foram testados em bases de dados comumente usadas para este tipo de problema. 


